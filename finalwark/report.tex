\documentclass[12pt,a4paper]{ctexart}


% 数学符号和公式
\usepackage{amsmath}
\usepackage{amssymb}
\usepackage{amsthm}

% 代码高亮
\usepackage{listings}
\usepackage{xcolor}

% VS Code 风格配色(参考截图:Light 主题)
\definecolor{vscodeBg}{HTML}{FFFFFF}
\definecolor{vscodeFg}{HTML}{000000}
\definecolor{vscodeKeyword}{HTML}{0000FF}
\definecolor{vscodeString}{HTML}{A31515}
\definecolor{vscodeComment}{HTML}{008000}
\definecolor{vscodeBuiltin}{HTML}{795E26}
\definecolor{vscodeConst}{HTML}{AF00DB}
\definecolor{vscodeOp}{HTML}{000000}
\definecolor{vscodeNumberLit}{HTML}{098658}
\definecolor{vscodeNumber}{HTML}{808080}
\definecolor{vscodeFrame}{HTML}{D0D0D0}

% 图形和表格
\usepackage{graphicx}
\usepackage{float}
\usepackage{booktabs}

% 页面设置
\usepackage{geometry}
\geometry{left=2.5cm,right=2.5cm,top=2.5cm,bottom=2.5cm}

% 超链接
\usepackage{hyperref}
\hypersetup{
    colorlinks=true,
    linkcolor=black,
    citecolor=blue,
    urlcolor=blue
}

% 算法伪代码
\usepackage{algorithm}
\usepackage{algorithmic}
\usepackage{tikz}
\usetikzlibrary{positioning}

% Python代码样式设置
\lstset{
    language=Python,
    basicstyle=\ttfamily\small\color{vscodeFg},
    keywordstyle=\bfseries\color{vscodeKeyword},
    commentstyle=\itshape\color{vscodeComment},
    stringstyle=\color{vscodeString},
    alsoletter={._},
    emphstyle=\color{vscodeBuiltin},
    % 常见内置函数/类型(规则增强,不是语义高亮)
    emph={print,sum,len,range,int,float,str,list,dict,set,tuple,enumerate,zip,map,filter,sorted,min,max,abs,all,any,open,input},
    % 常见常量/关键内置值
    emph={[2]True,False,None},
    emphstyle={[2]\bfseries\color{vscodeConst}},
    numbers=left,
    numberstyle=\tiny\color{vscodeNumber},
    stepnumber=1,
    numbersep=8pt,
    backgroundcolor=\color{vscodeBg},
    showspaces=false,
    showstringspaces=false,
    showtabs=false,
    frame=single,
    rulecolor=\color{vscodeFrame},
    tabsize=4,
    captionpos=b,
    breaklines=true,
    breakatwhitespace=false,
    upquote=true,
    columns=fullflexible,
    keepspaces=true,
    % 数字/常用运算符/点号做“规则染色”,让层次更像编辑器
    literate=%
        {0}{{\textcolor{vscodeNumberLit}{0}}}{1}
        {1}{{\textcolor{vscodeNumberLit}{1}}}{1}
        {2}{{\textcolor{vscodeNumberLit}{2}}}{1}
        {3}{{\textcolor{vscodeNumberLit}{3}}}{1}
        {4}{{\textcolor{vscodeNumberLit}{4}}}{1}
        {5}{{\textcolor{vscodeNumberLit}{5}}}{1}
        {6}{{\textcolor{vscodeNumberLit}{6}}}{1}
        {7}{{\textcolor{vscodeNumberLit}{7}}}{1}
        {8}{{\textcolor{vscodeNumberLit}{8}}}{1}
        {9}{{\textcolor{vscodeNumberLit}{9}}}{1}
        {.}{{\textcolor{vscodeOp}{.}}}{1}
        {=}{{\textcolor{vscodeOp}{=}}}{1}
        {+}{{\textcolor{vscodeOp}{+}}}{1}
        {-}{{\textcolor{vscodeOp}{-}}}{1}
        {*}{{\textcolor{vscodeOp}{*}}}{1}
        {/}{{\textcolor{vscodeOp}{/}}}{1}
        {<}{{\textcolor{vscodeOp}{<}}}{1}
        {>}{{\textcolor{vscodeOp}{>}}}{1}
        {(}{{\textcolor{vscodeOp}{(}}}{1}
        {)}{{\textcolor{vscodeOp}{)}}}{1}
        {[}{{\textcolor{vscodeOp}{[}}}{1}
        {]}{{\textcolor{vscodeOp}{]}}}{1}
        {:}{{\textcolor{vscodeOp}{:}}}{1}
        {,}{{\textcolor{vscodeOp}{,}}}{1}
        ,
    escapeinside={\%*}{*)},
    xleftmargin=2em,
    xrightmargin=2em,
    aboveskip=1em
}

% 标题信息
\title{\textbf{背包问题求解算法报告}}
\author{姓名:孙智诚\quad 张文韬 \\ 学号:2451503\quad2451500}
\date{\today}

\begin{document}

\maketitle


\section{问题描述}

给定 $n$ 个物品,每个物品有体积 $W_i$ $(i=1,2,\ldots,n)$,
以及一个容量为 $T$ 的背包。问题的目标是找出能够恰好装满背包的物品组合方案。


\subsection{问题分析}

这是一个经典的组合优化问题,具有以下特点:

\begin{itemize}
    \item 每个物品只能被选择一次
    \item 选择的顺序不影响最终结果
    \item 只关心解的存在性,不要求最优解法
\end{itemize}

从算法设计角度分析,该问题可以采用多种求解策略:

\paragraph{回溯法} 遍历$W$数组,当剩余容量无法容纳任何物品时及时回溯,直到寻找出能恰好装满
$T$的数字组合。

\paragraph{可达集合法} 构建可达集合$a$,用以容纳$W$中任意元素之间通过加法运算
能够达到的结果,最后将该集合与背包容量$T$比对,若$T \in a$,则说明容量为
$T$的背包可以被恰好装满。\\
\\
本报告中,主要采用回溯法实现,因为该方法能够记录所有的解法,且时间、空间复杂度较法二均更低,适合展示算法思想和搜索过程,而法二则
只做为该问题解决方法多样性的例证。

\section{算法设计}

\subsection{算法思想}

本算法采用回溯法求解背包问题的所有可行解。用显式栈 $S$ 记录当前已选择的物品索引序列,用 $remain\_T$ 表示背包剩余容量。搜索过程等价于在一棵解空间树上进行深度优先遍历:每向下一层表示尝试将某个物品加入当前组合;当加入后仍满足 $remain\_T \ge 0$ 时继续扩展;若出现 $remain\_T < 0$(容量超出)或已无法继续选择物品,则当前分支不可行,需要回溯到上一层。

回溯时通过弹出栈顶元素撤销最近一次选择,并恢复相应的剩余容量,然后从被撤销物品的下一个编号继续尝试,从而避免重复枚举相同组合;当 $remain\_T = 0$ 时得到一组恰好装满背包的解并输出。最终当栈为空且不存在可继续扩展的分支时,说明所有可能组合均已遍历完成,算法结束。
\subsection{数据结构}

本算法使用的核心数据结构与状态变量如下:

\begin{itemize}
    \item \textbf{物品体积表 $W$:} 用一维数组存储各物品体积。实现中采用 1 开始的下标(令 $W[0]=0$ 作为占位),便于与题目编号对应。
    \item \textbf{显式栈 $S$:} 使用 Python 列表模拟栈结构,保存\emph{当前已选物品的索引序列}。入栈操作对应“选择该物品并向下一层搜索”(\texttt{append}),出栈操作对应“撤销最近一次选择并回溯”(\texttt{pop})。因此,$S$ 即为一条从根到当前节点的搜索路径。
    \item \textbf{剩余容量 $remain\_T$:} 记录当前路径下背包尚未填满的容量。每次选择物品 $i$ 时执行 $remain\_T\leftarrow remain\_T-W_i$;回溯撤销物品时执行 $remain\_T\leftarrow remain\_T+W_i$。
    \item \textbf{当前尝试指针 $i$:} 表示下一步准备尝试的物品编号。回溯时将 $i$ 更新为“刚撤销物品的下一个编号”($i\leftarrow top\_item+1$),从而保证每个物品最多被选择一次,并避免重复枚举相同组合。
    \item \textbf{计数器 $solution\_count$:} 用于统计并编号输出的可行解数量。
\end{itemize}


通过以上结构,递归回溯过程被等价地实现为栈的压入或弹出与状态变量更新,无需函数递归调用栈。

\subsection{算法伪代码}

\begin{algorithm}[H]
\caption{背包问题回溯求解}
\begin{algorithmic}[1]
\STATE 初始化:$S \leftarrow \emptyset$, $i \leftarrow 1$, $remain\_T \leftarrow T$
\WHILE{true}
    \WHILE{$i \leq n$ \AND $remain\_T - W[i] \geq 0$}
        \STATE $S \leftarrow S \cup \{i\}$
        \STATE $remain\_T \leftarrow remain\_T - W[i]$
        \IF{$remain\_T = 0$}
            \STATE 输出解 $S$
            \STATE 回溯:$i \leftarrow S.pop() + 1$
            \STATE $remain\_T \leftarrow remain\_T + W[i-1]$
            \STATE continue
        \ENDIF
        \STATE $i \leftarrow i + 1$
    \ENDWHILE
    \IF{$S = \emptyset$}
        \RETURN
    \ENDIF
    \STATE 回溯:$i \leftarrow S.pop() + 1$
    \STATE $remain\_T \leftarrow remain\_T + W[i-1]$
\ENDWHILE
\end{algorithmic}
\end{algorithm}

\section{程序实现}

\subsection{Python实现}

以下是使用Python实现的部分代码:

\begin{lstlisting}[caption={背包问题求解程序}]
def solve_package_all_solutions(W, n, T):
    S = [] 
    i = 1  
    remain_T = T  
    solution_count = 0  
    
    while True:
        while i <= n and remain_T - W[i] >= 0:
            S.append(i) 
            remain_T -= W[i] 
            if remain_T == 0:
                solution_count += 1
                print(f"解 {solution_count}:物品序号 {S}")
                top_item = S.pop()
                remain_T += W[top_item]
                i = top_item + 1
                continue
            i += 1
        
        if not S: 
            return
        
        top_item = S.pop()
        remain_T += W[top_item]
        i = top_item + 1
\end{lstlisting}



\section{实验结果}

\subsection{测试用例1}

输入:\\
物品体积:$W = [2, 3, 5, 7]$
背包容量:$T = 10$\\

输出:\\
开始搜索所有可能的解...\\
解 1:物品序号 [1, 2, 3] | 体积 [2, 3, 5] | 总体积=10\\
解 2:物品序号 [2, 4] | 体积 [3, 7] | 总体积=10\\
搜索完成,共找到 2 个解。

\subsection{测试用例2}

输入:\\
物品体积:$W = [2, 3, 5]$\\
背包容量:$T = 8$\\


输出:\\
开始搜索所有可能的解...\\
解 1:物品序号 [2, 3] | 体积 [3, 5] | 总体积=8\\
搜索完成,共找到 1 个解。

\subsection{测试用例3(无解情况)}

输入:\\
物品体积:$W = [3, 5, 7]$\\背包容量:$T = 11$\\

输出:\\
开始搜索所有可能的解...\\
无解!无法恰好装满背包。

\section{复杂度分析}

\subsection{时间复杂度}

在最坏情况下,算法需要遍历所有可能的物品组合。设物品数量为 $n$,背包容量为 $T$,最小物品体积为 $W_{min}$,则搜索树的最大深度为 $O(T/W_{min})$
,每个节点最多有 $n$ 个分支
,时间复杂度为 $O(n^{T/W_{min}})$,为指数级
\subsection{空间复杂度}

空间复杂度主要由显式栈决定:
\begin{equation}
    S(n) = O(T/W_{min})
\end{equation}
\section*{可达集合法实现}

除回溯法外,我们还给出了“可达集合法”的一种简单实现,用于判断背包容量 $T$ 是否存在可行组合。

\paragraph{思想} 提出一个可达集合 $a$,其中每个元素表示当前遍历过的元素通过加法能达到的所有值。初始时 $a=\{0\}$。当遍历到一个物品体积 $i$ 时,令
\[
    d = \{x+i\mid x\in a\}
\]
并将新得到的体积并入 $a$。若在更新过程中发现 $T\in a$,即可提前结束并判定背包恰好能被装满。


\section{评价和改进}

\subsection{评价}

本报告实现了基于回溯法的背包问题求解算法,能够找出所有满足条件的物品组合方案。算法思路还算清晰,实现简洁,适用于求解小规模的背包问题。

\subsection{优化方向}

尽管回溯法能够枚举所有可行解并直观体现搜索过程,但其缺点也较为明显:在最坏情况下需要遍历接近指数规模的解空间,随着物品数量增加运行时间会急剧上升;同时输出全部解的数量本身可能非常大,I/O 开销也会成为瓶颈。此外,若剪枝条件较弱或物品顺序不合适,会产生大量无效分支搜索与重复回溯。

针对上述问题,首先可以从\textbf{剪枝与预处理}入手提升搜索效率,例如先对物品体积进行排序,使得“不可能放下”的情况更早暴露;在搜索过程中,当 $remain\_T$ 已小于后续物品的最小体积时可直接回溯;或利用“后续物品体积总和不足以填满 $remain\_T$”等上界/下界条件,提前剪去必然无解的分支,从而显著减少无效探索。

可以引入记忆来避免重复搜索,回溯过程中常会多次遇到相同状态,例如到达同一物品编号 $i$ 且剩余容量同为 $remain\_T$ 的情形;若该状态已被证实无法得到解,则再次遇到时可直接跳过,从而减少大量重复回溯。
\section*{附录}
\begin{lstlisting}[caption={回溯法求解程序}]
def solve_package_all_solutions(W, n, T):

    S = [] 
    i = 1  
    remain_T = T  
    solution_count = 0  

    print("开始搜索所有可能的解...")

    while True:
        while i <= n and remain_T - W[i] >= 0:
            S.append(i) 
            remain_T -= W[i] 
            if remain_T == 0:
                solution_count += 1
                print(f"解 {solution_count}:物品序号 {S}", end="")
                volumes = [W[idx] for idx in S]
                print(f" | 体积 {volumes} | 总体积={sum(volumes)}")


                top_item = S.pop()
                remain_T += W[top_item]
                i = top_item + 1
                continue

            i += 1

        if not S: 
            if solution_count == 0:
                print("无解!无法恰好装满背包。")
            else:
                print(f"搜索完成,共找到 {solution_count} 个解。")
            return


        top_item = S.pop()
        remain_T += W[top_item]
        i = top_item + 1 




if __name__ == "__main__":

    print("=" * 70)
    print("测试用例1".center(70))
    print("=" * 70)
    W1 = [0, 2, 3, 5, 7] 
    n1 = 4
    T1 = 10

    print(f"物品体积:{W1[1:n1+1]}")
    print(f"背包容量:{T1}\n")


    solve_package_all_solutions(W1, n1, T1)


    print("\n" + "=" * 70)
    print("测试用例2".center(70))
    print("=" * 70)
    W2 = [0, 2, 3, 5]
    n2 = 3
    T2 = 8

    print(f"物品体积:{W2[1:n2+1]}")
    print(f"背包容量:{T2}\n")

    solve_package_all_solutions(W2, n2, T2)

    #无解的情况
    print("\n" + "=" * 70)
    print("测试用例3 (无解)".center(70))
    print("=" * 70)
    W3 = [0, 3, 5, 7]
    n3 = 3
    T3 = 11

    print(f"物品体积:{W3[1:n3+1]}")
    print(f"背包容量:{T3}\n")

    solve_package_all_solutions(W3, n3, T3)

\end{lstlisting}
\begin{lstlisting}[caption={可达集合法求解程序}]
    def neng_zhuang_man_bu(W, T):
    a = [0] #用来装"可以通过W内数字相加达到的数字"的数组,初始为0,这是不证自明的
    for i in W:
        if i > T:
            continue
        d = [x + i for x in a] #为a中每个元素都加上本次遍历到的数字
        a = a + d #把新数组与旧数组合并
        if T in a:
            print(f"体积{T}刚好能被填满")
            return True

    print(f"体积{T}没法被{W}刚好填满")
    return False


if __name__ == "__main__":
    # 测试示例
    W = [2, 3, 5]
    T = 8
    neng_zhuang_man_bu(W, T)
\end{lstlisting}
\end{document}
